% ======================================================================

% ======================================================================

\documentclass[12pt,geral,titlewithdate]{uftdocs}

\address{109 Norte, Avenida NS 15, ALCNO 14, Bloco Bala 2 - Sala 21}
\cep{77402-970}
\phone{(63) 3232-8027}
\mail{comppalmas@uft.edu.br}
\department{ciência da computação}
\city{Palmas}

\title{Normativa para elaboração de Projeto de Graduação}
\date{12}{11}{2016}
\descricao{}
\descricao{Dispõe sobre as atividades relacionadas com as disciplinas de Projeto de Graduação I e II do Curso de Bacharelado em Ciência da Computação, do campus universitário de Palmas, da Fundação Universidade Federal do Tocantins.}

\begin{document}

\maketitle

%\linenumbers

\chapter{Das Disposições Preliminares}
\label{chap:disposicoes}


\artigo O Projeto de Graduação é elaborado nas disciplinas de Trabalho de Conclusão de Curso I e Trabalho de Conclusão de Curso II, do curso de Bacharelado em Ciência da Computação, referente ao último Projeto Pedagógico do Curso vigente.

\artigo São objetivos do Projeto de Graduação:

\begin{easylist}\ListProperties(Start1=1)
# Desenvolver a capacidade de aplicação dos conceitos e teorias adquiridas durante o curso de forma integrada, por meio da execução de um projeto de pesquisa.
# Desenvolver a capacidade de planejamento e disciplina para resolver problemas dentro das diversas áreas de formação.
# Despertar o interesse pela pesquisa como meio para a resolução de problemas.
# Estimular o espírito empreendedor, por meio da execução de projetos que levem ao desenvolvimento de produtos, os quais possam ser patenteados e/ou comercializados.
# Intensificar a extensão universitária, por intermédio da resolução de problemas existentes nos diversos setores da sociedade.
# Estimular a construção do conhecimento coletivo.
# Estimular a interdisciplinaridade.
# Estimular a inovação tecnológica.
# Estimular o espírito crítico e reflexivo no meio social onde está inserido.
# Estimular a formação continuada.
\end{easylist}

\artigo O Projeto de Graduação deverá ser desenvolvido individualmente. 

\begin{paragrafos}

\paragrafo O Projeto de Graduação será caracterizado por uma pesquisa científica e/ou tecnológica aplicada.

\paragrafo É vedada a convalidação de Projeto de Graduação realizado em outro curso de graduação.

\end{paragrafos}

\artigo O tema do Projeto de Graduação deve estar relacionado a uma das áreas de abrangência do curso, de acordo com a classificação de áreas da {\it Association for Computing Machinery} (ACM).

\artigo O Projeto de Graduação será deliberado por uma Comissão de Projeto de Graduação, que será composta pelos Professores da disciplinas de Trabalho de Conclusão de Curso I e Trabalho de Conclusão de Curso II, coordenador de curso e por, até mais 2 (dois) pareceristas, sendo estes professores efetivos do Curso, designados pelo Colegiado.

\chapter{Da Matrícula}
\label{chap:tecnicas}

\artigo Para se matricular na disciplina de Projeto de Graduação, o aluno deverá ter cumprido os pré-requisitos estabelecidos no Projeto Pedagógico do Curso (PPC).

\begin{paragrafos}

\paragrafo A matrícula na disciplina de Projeto de Graduação concede ao aluno o direito de elaborar e defender seu Projeto de Graduação, conforme calendário estabelecido pela Comissão de Projeto de Graduação, desde que cumpridos os deveres.

%\paragrafo Para a autorização da matrícula, é necessária a definição do tema do projeto e do orientador responsável. Estas informações devem ser fornecidas à Comissão de Projeto de Graduação através de um formulário específico devidamente assinado, pelo aluno e pelo(s) orientador(es), e entregue na Secretaria do Curso, durante o período de matrícula.

\paragrafo No formulário também deve ser definido o Plano de Trabalho a ser realizado.

\end{paragrafos}

\artigo  A elaboração do Plano de Trabalho devem seguir o modelo aprovado pela Comissão de Projeto de Graduação. 

\chapter{Do Acompanhamento}
\label{chap:tecnicas}

\artigo A orientação do aluno no Projeto de Graduação deverá ser efetuada por um docente do curso de Bacharelado em Ciência da Computação da Universidade Federal do Tocantins, Campus Universitário de Palmas - TO (excluindo-se professores temporários e substitutos).

{\bf Parágrafo único}. Caberá ao aluno a escolha do orientador e, junto com o mesmo, a definição do tema do Projeto de Graduação. Ao orientador escolhido pelo aluno é facultado aceitar ou rejeitar o convite para a orientação.

\artigo O Plano de Trabalho entregue à Comissão de Projeto de Graduação será utilizado para acompanhamento do trabalho.

\begin{paragrafos}
\paragrafo A proposta de que trata este artigo deverá ser revisada e assinada pelo orientador do aluno antes de ser entregue à Comissão de Projeto de Graduação.

\paragrafo A proposta será avaliada pela Comissão de Projeto de Graduação.

\paragrafo O desenvolvimento do trabalho só se dará a partir da aprovação do Plano de Trabalho.

\end{paragrafos}

\artigo Caso o aluno não mantenha desempenho satisfatório no desenvolvimento do Projeto de Graduação, o orientador poderá solicitar sua interrupção.

\begin{paragrafos}
\paragrafo A solicitação de interrupção do Projeto de Graduação deverá ser comunicada de forma escrita, e justificada pelo orientador, para a Comissão de Projeto de Graduação, que dará ciência ao aluno.

\paragrafo Caberá ao aluno a escolha de um novo orientador, a adequação do tema, a elaboração e a entrega da nova Proposta de Trabalho durante o período de alteração de matrícula. A nova proposta ficará sujeita à aprovação da Comissão de Projeto de Graduação.

\end{paragrafos}

\artigo Durante o desenvolvimento do trabalho poderá ocorrer substituição do orientador, desde que justificada e comunicada por escrito à Comissão de Projeto de Graduação.

\begin{paragrafos}
\paragrafo A justificativa deve ter a anuência dos dois orientadores envolvidos: o
anterior e o novo. 

\paragrafo Caberá ao aluno a adequação do tema do Projeto de Graduação, a elaboração e a entrega da nova Proposta de Trabalho em até 45 (quarenta e cinco) dias após o início do semestre letivo, no máximo, e ficará sujeita à aprovação pela Comissão de Projeto de Graduação.

\end{paragrafos}

\artigo O aluno poderá contar com a co-orientação de profissionais da área de Computação, desde que haja a anuência do orientador, que continuará sendo o responsável pela orientação do Projeto de Graduação.

{\bf Parágrafo único}. O co-orientador deve ser um docente ou um profissional com comprovada competência no tema do Projeto de Graduação.

\artigo A Comissão de Projeto de Graduação deverá elaborar cronogramas das atividades envolvidas no Projeto de Graduação.

\chapter{Da Avaliação da Proposta de Trabalho}

\artigo A Proposta de Trabalho deverá ser entregue para avaliação das disciplinas de Trabalho de Conclusão de Curso I e Trabalho de Conclusão de Curso II.

\artigo A Proposta de Trabalho será avaliada pela Comissão do Projeto de Graduação. O resultado da avaliação da Proposta de Trabalho deverá ser divulgado aos alunos em até 15 dias, no máximo, após a data de entrega à Comissão de Projeto de Graduação.

\artigo\label{cri} A Proposta de Trabalho deverá ser avaliada observando-se a relevância e pertinência do tema proposto, com base nos seguintes critérios:

\begin{easylist}\ListProperties(Start1=1)
# valor acadêmico, emprego de tecnologias atuais e utilidade do projeto;
# viabilidade técnico/científica do trabalho;
# clareza na apresentação da proposta;
# adequação do cronograma de atividades (que deverá obedecer o calendário da disciplina.
\end{easylist}

\artigo O aluno cuja Proposta de Trabalho não for aprovada terá um prazo adicional de 15 (quinze) dias corridos para apresentação de nova proposta ou para readequação da proposta inicial. No caso de readequação da proposta inicial, as recomendações exaradas pela Comissão de Projeto de Graduação deverão ser atendidas.

{\bf Parágrafo único}. Caso o aluno não apresente nova proposta ou não faça as alterações na proposta inicial, ficará impossibilitado de prosseguir com as atividades exigidas pela disciplina e, portanto, estará reprovado.

\artigo A avaliação da Proposta de Trabalho será realizada conforme a Ficha de Avaliação da Proposta de Trabalho de Conclusão de Curso, sugerida pela Comissão de Projeto de Graduação e aprovada pelo Colegiado do Curso.


{\bf Parágrafo único}. A Ficha de Avaliação da Proposta de Trabalho de Conclusão de Curso deverá conter um campo para atribuição de nota.

\artigo O aluno reprovado na disciplina que pretende manter o mesmo projeto aprovado em ano anterior deverá apresentar nova Proposta de Trabalho, identificando quais atividades já foram finalizadas e quais atividades ainda deverão ser executadas. 

\chapter{Do Relatório de Acompanhamento}

\artigo O aluno matriculado na disciplina de Trabalho de Conclusão de Curso I, deverá entregar até 30 (trinta) dias antes do término do semestre letivo o relatório de acompanhamento.

{\bf Parágrafo único}. O aluno e seu orientador, com base no trabalho desenvolvido, deverão encaminhar um relatório de acompanhamento à Comissão de Projeto de Graduação.

\artigo Os objetivos do relatório são:

\begin{easylist}\ListProperties(Start1=1)
# forçar o relato do aluno ao orientador;
# permitir que o orientador possa cumprir seu papel, sugerindo eventuais mudanças de rumo;
# permitir ao orientador identificar situações tais como aquelas de trabalhos que não estejam sendo realizados.
\end{easylist}

\artigo O relatório de acompanhamento deverá ser entregue em modelo elaborado pela Comissão de Projeto de Graduação, e onde serão apresentados o resultado de seu trabalho e o planejamento das atividades a serem desenvolvidas na disciplina de Trabalho de Conclusão de Curso II.

\artigo Junto com a entrega do relatório de acompanhamento, o professor orientador deverá encaminhar a sua avaliação do trabalho.

\artigo Os documentos de avaliação e acompanhamento do trabalho serão encaminhados para a Comissão de Projeto de Graduação, para avaliação do trabalho.

\artigo O orientador, a Comissão de Projeto de Graduação ou o próprio aluno poderão solicitar um encontro, mediante formulário próprio, onde o aluno poderá apresentar o trabalho. 

{\bf Parágrafo único}. A forma (apresentação de slides, entrevista, reunião, prova etc.), banca examinadora, data e local dessa apresentação serão determinados pela Comissão de Projeto de Graduação.

\chapter{Da Banca Examinadora}

\artigo A banca examinadora deverá ser solicitada, mentiante formulário próprio, pelo orientador 30 (trinta) dias até o fim do semestre letivo.

\artigo A banca examinadora do Projeto de Graduação será composta por três membros da seguinte forma:

\begin{easylist}\ListProperties(Start1=1)
# o orientador ou o co-orientador do aluno, que preside a banca;
# um professor do Curso de Ciência da Computação, CUP, com experiência na área de pesquisa do trabalho;
# um professor da UFT ou de outra Instituição de Ensino Superior, ou um
profissional com conhecimento reconhecido sobre o tema do trabalho, indicado pelo orientador. Na ausência da indicação do orientador no prazo estipulado pelo calendário da disciplina, a Comissão de Projeto de Graduação fará a indicação.
\end{easylist}

\begin{paragrafos}
\paragrafo Será vedada a participação do orientador e do co-orientador concomitantemente na banca examinadora.

\paragrafo A banca deverá ser aprovada pela Comissão de Projeto de Graduação;

\end{paragrafos}

\artigo Em caso do não comparecimento do orientador ou co-orientador, se houver, o orientador deverá informar por escrito, segundo Formulário Próprio, à Comissão do Projeto de Graduação, e indicar substituto em até 5 (cinco) dias antes da defesa do trabalho, no mínimo.

{\bf Parágrafo único}. Em caso de não informe da ausência do orientador, o aluno estará impossibilitado de apresentar o trabalho.

\artigo A banca examinadora fará a avaliação do trabalho de conclusão de curso de acordo com a Ficha de Avaliação do Trabalho de Conclusão de Curso. A Ficha deverá ser aprovada pela Comissão de Projeto de Graduação.

\artigo Para aprovação, o aluno deve obter nota final igual ou superior a 5,0 (cinco).

\artigo O aluno terá um prazo de até 7 (sete) dias úteis para realizar as correções solicitadas e entregar a versão final do Projeto de Graduação para conferência da banca examinadora.


\artigo Caso seja constatada a existência de plágio, a banca deverá registrar o ocorrido na Ficha de Avaliação do Trabalho de Conclusão de Curso, e a Comissão de Projeto de Graduação deverá encaminhar o assunto ao Colegiado de Curso de Ciência da Computação para providências. 

\chapter{Da Apresentação do Trabalho de Conclusão de Curso}

\artigo A apresentação do Trabalho de Conclusão de Curso será realizada perante uma banca examinadora, em data, horário e local informados aos alunos, com 7 dias de antecedência.

\begin{paragrafos}
\paragrafo A apresentação do Trabalho de Conclusão de Curso para a banca examinadora deverá ocorrer em tempo hábil para que se possa fazer a entrega da Versão Final da Monografia antes da Mostra de Trabalhos de Conclusão de Curso.

\paragrafo A Mostra dos Trabalhos deverá ocorrer  15 (quinze) dias antes da data limite para entrega de notas e frequências estabelecida no Calendário de Graduação Campus Universitário de Palmas.

\paragrafo A apresentação do Trabalho de Conclusão de Curso na Mostra de Trabalhos de Conclusão de Curso é obrigatória para o aluno ser aprovado na disciplina de Trabalho de Conclusão de Curso II.

\paragrafo O aluno matriculado em Trabalho de Conclusão de Curso II poderá solicitar a apresentação do trabalho fora da Mostra dos Trabalhos, em qualquer data dentro do semestre letivo, o qual será avaliado pela Comissão de Projeto de Graduação.

\paragrafo O aluno terá de 20 (vinte) à 30 (trinta) minutos para apresentação do trabalho.

\paragrafo A apresentação do trabalho é pública, salvo exceções em que o aluno se sinta desconfortável ou constrangido, com anuência do Núcleo de Apoio Psico-Pedagógico.

\end{paragrafos}

\artigo O aluno deverá entregar à Comissão de Projeto de Graduação, na data determinada no cronograma da disciplina:

\begin{easylist}\ListProperties(Start1=1)
# a versão final da monografia corrigida.
# um CD-ROM com todo o conteúdo do Projeto de Graduação, que inclui a Proposta do Trabalho, a Monografia, os arquivos das apresentações do trabalho e os códigos fonte e executáveis dos programas desenvolvidos, além das respectivas documentações.
#  a autorização de publicação assinada pelo Orientador, Formulário preenchido e assinado pelo aluno para publicação na Biblioteca e um CD-ROM somente com a monografia em formato PDF, sem restrições de escrita.
\end{easylist}

\artigo Para a apresentação do trabalho à banca examinadora, a Comissão de Projeto de Graduação deverá entregar uma cópia da monografia para cada membro da banca com, no mínimo, 10 dias de antecedência.

\artigo Se houver sugestão da Banca examinadora para correções na monografia, o aluno deverá entregar uma nova versão corrigida por ele e revisada pelo orientador em até 10 dias corridos, contados a partir da apresentação do Trabalho de Conclusão de Curso. Se esta exigência não for cumprida, o aluno será reprovado. 

\chapter{Dos Critérios de Avaliação}

\artigo O cálculo da Média Final do aluno na disciplina de Trabalho de Conclusão de Curso I será feito baseando-se:

\begin{easylist}\ListProperties(Start1=1)
# As notas variam no intervalo de 0 (zero) a 10 (dez),
# na nota atribuídas pela Comissão de Projeto de Graduação, numa proporção de 50\% e
# na nota do orientador, numa proporção de 50\%.
\end{easylist}

{\bf Parágrafo único}. Em caso de solicitação de Banca Examinadora os critérios de avaliação serão os mesmo do artigo \ref{cri}.
 
\artigo\label{cri} O cálculo da Média Final do aluno na disciplina Trabalho de Conclusão de Curso II será feito baseando-se:

\begin{paragrafos}
\paragrafo As notas variam no intervalo de 0 (zero) a 10 (dez), sendo atribuída uma nota por cada avaliador e Média Final será a média aritmética entre as notas.

\paragrafo A banca examinadora avaliará a qualidade da Monografia e da apresentação oral do Trabalho de Conclusão de Curso realizadas pelo aluno, de acordo com os critérios:

\begin{easylist}\ListProperties(Start1=1)
# valor acadêmico, emprego de tecnologias atuais e utilidade do projeto;
# viabilidade técnico/científica do trabalho;
# clareza na apresentação oral;
# uso adequado da metodologia de científica e; padrões de projetos e métricas de qualidade, quando for o caso;
# uso adequado da escrita em norma culta.
\end{easylist}

\paragrafo A avaliação da Banca Examinadora deverá ser feita com o auxílio de uma Ficha de Avaliação, que deverá ser aprovada pelo Comissão de Projeto de Graduação.

\paragrafo O resultado da avaliação, Média Final, séra divulgado posteriormente à data da avaliação por meio eletrônico próprio de divulgação.
 
\end{paragrafos}

\chapter{Do Artigo Completo como Projeto de Graduação}

\artigo Artigo completo será válido como Projeto de Graduação somente se aceito em congresso ou periódico reconhecido pelo Colegiado do Curso e o aluno deve ter cumprido, no mínimo, 50\% (cinquenta por cento) dos créditos pertencentes ao currículo pleno do curso quando do aceite do artigo.

\artigo A documentação comprobatória relativa ao aceite do artigo deverá ser entregue à Comissão de Projeto de Graduação pelo orientador constando a assinatura do mesmo. 

\begin{paragrafos}
\paragrafo O aluno deve ser, obrigatoriamente, o primeiro autor;

\paragrafo O orientador do Projeto de Graduação deve ser coautor do artigo.

\paragrafo A Comissão de Projeto de Graduação fará a deliberação sobre a admissibilidade do artigo como Projeto de Graduação, levando-se em consideração o artigo \ref{cri}.
 
\end{paragrafos}

\chapter{Do Plano de Negócios como Projeto de Graduação}

\artigo O Plano de Negócios será válido como Projeto de Graduação somente se obtiver como resultado um produto empreendedor e/ou de inovação tecnológica e o aluno deve ter cumprido, no mínimo, 50\% (cinquenta por cento) dos créditos pertencentes ao currículo pleno do curso quando do aceite do artigo.

{\bf Parágrafo único}. O Plano de Negócios será válido como Projeto de Graduação somente se previamente aprovado pelos orgãos de comprovada competência para admissibilidade do Plano. Tais como: SEBRAE, SENAI e FEIS.

\artigo A documentação comprobatória relativa ao aceite do Plano de Negócios deverá ser entregue à Comissão de Projeto de Graduação pelo orientador constando a assinatura do mesmo. 

\begin{paragrafos}
\paragrafo O aluno deve ser, obrigatoriamente, o primeiro autor;

\paragrafo O orientador do Projeto de Graduação deve ser coautor do Plano de Negócios.

\paragrafo A Comissão de Projeto de Graduação fará a deliberação sobre a admissibilidade do artigo como Projeto de Graduação, levando-se em consideração o artigo \ref{cri}.
 
\end{paragrafos}


\chapter{Atribuições aos Envolvidos no Projeto de Graduação}

\artigo A orientação do Projeto de Graduação, entendida como processo de acompanhamento didático-pedagógico é de responsabilidade dos docentes do curso Ciência da Computação -- Campus de Palmas.

\artigo Cada orientador pode orientar no máximo 4 (quatro) alunos.

\artigo Cabe ao Colegiado do Curso de Ciência da Computação:

\begin{easylist}\ListProperties(Start1=1)
# aprovar os modelos para elaboração da Monografia;
# analisar e aprovar a Ficha de Avaliação de Proposta de Trabalho;
# analisar e aprovar a Ficha de Avaliação do Trabalho de Conclusão de Curso I e II;
# aprovar os Cronogramas de aulas propostas pela Comissão de Projeto de Graduação, antes do início do período letivo das disciplinas;
# manifestar-se em casos de plágio; 
\end{easylist}

\artigo Cabe à Comissão de Projeto de Graduação:

\begin{easylist}\ListProperties(Start1=1)
# analisar e aprovar as bancas examinadoras;
# fazer a divulgação das bancas examinadoras;
# emitir os certificados de orientação, co-orientação e de participação na banca examinadora;
# encaminhar para o Colegiado do Curso todos os documentos necessários à deliberação;
# publicar na página oficial da disciplina, ou outro meio adequado, as datas pertinentes ao cronograma da disciplina.
# disponibilizar os modelos de documentos a serem entregues pelos alunos;
# fornecer informações sobre o Projeto de Graduação aos orientadores e alunos;
# recolher os documentos elaborados pelos alunos e orientadores durante o desenvolvimento das atividades, conforme cronograma da disciplina;
# avaliar a proposta de trabalho, nos termos e critérios estabelecidos neste regulamento;
# indicar a participação dos membros da banca, caso o orientador não o faça, dando preferência aos professores com experiência na área de pesquisa do trabalho;
# definir locais, datas e horários para realização das bancas examinadoras;
# entregar para cada membro da banca examinadora uma cópia da monografia.
# providenciar informações aos membros das bancas examinadoras em relação aos procedimentos referentes à avaliação dos alunos;
# elaborar normas e procedimentos administrativos destinados a aprimorar as atividades do trabalho de conclusão de curso;
# avaliar e encaminhar propostas de alteração deste Regulamento, com base em experiências acumuladas no decorrer do curso, sugestões de orientadores, membros de bancas examinadoras e alunos formandos, ou então, readequá-lo para atender às resoluções da universidade;
# zelar pela observância do presente Regulamento, comunicando problemas e irregularidades ao Colegiado do curso;
# servir de mediador, em caso de ocorrência de conflitos de interesses, envolvendo alunos e orientadores no decorrer do trabalho;
# assessorar os alunos na resolução de assuntos pertinentes ao Projeto de Graduação;
# a Comissão de Projeto de Graduação deverá promover atividades visando auxiliar os alunos da turma de formandos do ano seguinte do curso de Ciência da Computação a identificar possíveis orientadores e temas para os seus Trabalhos de Conclusão de Curso.
\end{easylist}

\artigo Compete ao orientador do Trabalho de Conclusão de Curso:

\begin{easylist}\ListProperties(Start1=1)
# orientar os alunos nas questões relacionadas ao conteúdo, forma e sequência;
# indicar materiais de referencial teórico como obras bibliográficas e periódicos, especificando, quando convier, os capítulos, as páginas e os artigos mais significativos ao trabalho a ser desenvolvido;
# orientar e corrigir os documentos exigidos pela disciplina e que devem ser elaborados por seus orientandos, com sua anuência;
# orientar os alunos na preparação da apresentação, justificativa e apresentação do trabalho;
# estimular o orientando para que a versão final do trabalho demonstre as competências e habilidades adquiridas, e que seja produzido em nível adequado a cursos de graduação e, ainda, que obedeça às normas técnicas estabelecidas;
# comunicar por escrito a Comissão de Projeto de Graduação sobre os problemas relacionados ao trabalho, caso sejam constatados negligências do aluno, despreparo ou falta de completude do trabalho;
# justificar e comunicar por escrito a Comissão de Projeto de Graduação caso o trabalho for interrompido;
# exigir do aluno o cumprimento dos prazos limites estipulados no cronograma da disciplina;
# incentivar a elaboração de relatórios técnicos/notas didáticas sobre os assuntos pesquisados pelos alunos;
# estimular o envio de trabalhos técnicos para eventos da área, mesmo depois do aluno ter se formado;
# coordenar o trabalho desenvolvido pela banca examinadora, coletando os respectivos pareceres e notas;
# encaminhar à Comissão de Projeto de Graduação a nota final da Banca examinadora, assim como as Fichas de Avaliação utilizadas pela banca examinadora.
\end{easylist}

\artigo Compete ao co-orientador do Trabalho de Conclusão de Curso (quando
houver):

\begin{easylist}\ListProperties(Start1=1)
# auxiliar o orientador na orientação do trabalho. 
\end{easylist}

\artigo Compete ao orientando:

\begin{easylist}\ListProperties(Start1=1)
# cumprir o plano e cronograma estabelecidos em conjunto com o seu orientador e pela Comissão de Projeto de Graduação;
# contatar seu orientador regularmente, durante o período de desenvolvimento do trabalho;
# cumprir rigorosamente as datas de entrega de documentos, bem como, o cronograma das atividades da disciplina;
# empenhar-se na busca de conhecimento e assessoramento necessário ao desempenho das atividades do trabalho;
# entregar os documentos especificados para cada fase do trabalho, sempre com a anuência do orientador;
# comunicar, por escrito, à Comissão de Projeto de Graduação a necessidade de alterações na proposta de trabalho, encaminhando a proposta alterada, juntamente com as justificativas necessárias e a anuência do orientador;
# comunicar, formalmente, da desistência do trabalho, quando for o caso;
# comprometer-se para que seu trabalho seja fundamentado na autenticidade e legitimidade, assumindo na íntegra a autoria do trabalho elaborado passo a passo, segundo o cronograma proposto;
# comunicar, por escrito, à Comissão de Projeto de Graduação eventuais problemas relacionados à orientação;
# fazer a redação final do trabalho segundo as normas estabelecidas para trabalhos acadêmicos (Aprovado por Colegiado de Curso).
\end{easylist}

\artigo Cabe aos envolvidos no processo de desenvolvimento do trabalho:

\begin{easylist}\ListProperties(Start1=1)
# procurar manter o interesse pelas atividades desenvolvidas;
# primar pela qualidade durante todo o processo;
# agir com integridade;
# informar-se sobre as normas e regulamentos do Projeto de Graduação;
# cumprir as normas e regulamentos do Projeto de Graduação.
\end{easylist}

{\bf Parágrafo único}. A orientação é de interesse do aluno e, portanto deve partir deste a iniciativa de procurar seu orientador e co-orientador, caso exista, sob pena de caracterizar o não comprometimento do aluno com o processo. 

\chapter{Das Disposições Gerais}

\artigo Os casos omissos serão resolvidos pela Comissão de Projeto de Graduação, no âmbito de suas competências. 

\addtocounter{artigo}{100}


\end{document}


